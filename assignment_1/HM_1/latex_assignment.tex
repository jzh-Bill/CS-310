\documentclass[11pt]{article}
\usepackage[paper=letterpaper,margin=1in]{geometry}
\usepackage[parfill]{parskip}
\usepackage{amsmath}
\usepackage{graphicx}
\usepackage{fancyvrb}
\usepackage{upquote}
\usepackage[en-GB]{datetime2}
\DTMlangsetup[en-GB]{ord=omit}

\newcommand{\problem}[1]{\vspace*{2ex}\textbf{Problem #1 ---} }
\newcommand{\answer}{\textit{Answer: } }

\begin{document}
\thispagestyle{empty}

\begin{center}
{\large CS 310}\\
Assignment 0820\\
\today
\end{center}
\begin{flushright}
\emph{Zhihao Jin(Bill)}
\end{flushright}

\problem{1} Algorithms typically have the following properties:
\begin{itemize}
\item the steps are stated \emph{unambiguously} so that there is
  no question how a step is performed
\item the algorithm is \emph{deterministic} so that repeating the
  algorithm on the same input produces the same output
\item the algorithm is \emph{finite} because it terminates after a
  finite number of steps have been performed
\item the algorithm produces \emph{correct} output for a given input
\end{itemize}

For the following algorithm, for each property listed above, determine
whether the algorithm exhibits this property:

\begin{Verbatim}[numbers=left,xleftmargin=5mm]
 unsigned max3(unsigned a, unsigned b, unsigned c)
 {
   unsigned result = a;
   if (b > result)
   {
     result = b;
   }
   if (c > result)
   {
     result = c;
   }
   return result;
 }
\end{Verbatim}

\answer This algorithm is unambiguous because the syntax for the
operations is well-understood.  It is deterministic because it always
produces the same output for a given input.  It is finite because the
number of lines of code executed (including the header) is strictly
between 3 and 7 inclusive.  It is correct because for all possible
combinations of input it produces the correct output.

\problem{2} Using the same properties as stated in problem 1, analyze
the following algorithm:

\begin{Verbatim}[numbers=left,xleftmargin=5mm]
  void shuffle(vector<unsigned>& a)
  {
    for (size_t index = 0; index < a.size(); index++)
    {
        swap(a.at(index), a.at(get_random_within(index, a.size()));
    }
  }
\end{Verbatim}

\answer This algorithm is unambiguous and quit easy to
understand. However, it is not deterministic since the output will not
be always same based on the random function. It is finite because the
number of iterations is from zero to the size of vector. But it
doesn't output anything, instead, swapping the position of the
elements in the vector.

\problem{3} What is the hexadecimal representation of $770_{10}$?

\answer The first few powers of 16 are:

\begin{align*}
  16^0 &= 1\\
  16^1 &= 16\\
  16^2 &= 256\\
  16^3 &= 4096\\
\end{align*}
Thus we have:
\begin{equation*}
  \begin{split}
    770&\\
    \underline{-3 \times 256 = 768}&\\
    2&\\
    \underline{-0 \times 16 = 0}&\\
    2&\\
    \underline{-2 \times 1 = 2}&\\
    0&
  \end{split}
\end{equation*}

And thus we have $770_{10} = 302_{16}$.\\
\problem{4} Based on your answer to the previous problem, what is the binary representation of $768_{10}$?

\answer The first few powers of 2 are:
\begin{align*}
 2^0 &= 1\\
 2^1 &= 2\\
 2^2 &= 4\\
 2^3 &= 8\\
 2^4 &= 16\\
 2^5 &= 32\\
 2^6 &= 64\\
 2^7 &= 128\\
 2^8 &= 256\\
 2^9 &= 512\\
 2^{10} &= 1024\\
\end{align*}

Thus we have:

\begin{equation*}
  \begin{split}
    768&\\
    \underline{-1 \times 512 = 512}&\\
    256&\\
    \underline{-1 \times 256 = 256}&\\
    0&\\
    \underline{-0 \times 128 = 0}&\\
    0&\\
    \underline{-0 \times 64 = 0}&\\
    0&\\
    \underline{-0 \times 32 = 0}&\\
    0&\\
    \underline{-0 \times 16 = 0}&\\
    0&\\
    \underline{-0 \times 8 = 0}&\\
    0&\\
    \underline{-0 \times 4 = 0}&\\
    0&\\
    \underline{-0 \times 2 = 0}&\\
    0&\\
    \underline{-0 \times 1 = 0}&\\
    0&
  \end{split}
\end{equation*}

And thus we have $768_{10} = 1100000000_{2}$.

\problem{5} What is the decimal representation of \texttt{0x1abc}?

\answer 
Based on the table below we can find that:\\

\begin{center}
 \begin{tabular}{||c c c c c c c c c c c c c c c c c||} 
 \hline
 Hex number: & 0 & 1 & 2 & 3 & 4 & 5 & 6 & 7 & 8 & 9 & A & B & C & D & E & F \\ [0.5ex] 
 \hline
 decimal number: & 0 & 1 & 2 & 3 & 4 & 5 & 6 & 7 & 8 & 9 & 10 & 11 & 12 & 13 & 14 & 15\\ 
 \hline    
 \end{tabular}
\end{center}

\begin{equation*}
    1 \times 16^3 + 10 \times 16^2 + 11 \times 16^1 + 12 \times 16^0 = 6844_{10}
\end{equation*}
And therefore, the decimal representation of \texttt{0x1abc} is 6844.


\end{document}



